\section{Methodology}\label{sec:Methodology}

O desenvolvimento do SISDOT deu-se no âmbito do projeto PROMISE-EB (Projeto de Pesquisa para Validação de Práticas e Métodos de Desenvolvimento de Software para o Exército Brasileiro). Entre as atividades deste projeto estava o desenvolvimento do SISDOT, a partir de um sistema já existente, escrito em Delphi e voltado para desktop.

Como não foi possível utilizar o sistema legado, foi realizada uma engenharia reversa a partir do manual de uso deste sistema, já que sistemas legados costumam ser uma das fontes de informação essenciais para a análise de domínio de aplicações. Engenharia de Domínio (ED) se preocupa com a identificação e modelagem de características comuns e variáveis em aplicações de um dado domínio, gerando como resultado modelos de domínio (e.g. casos de uso de domínio, classes do domínio), com o apoio oferecido por processos de ED que, geralmente, compreendem as fases de análise, projeto e implementação.

O desenvolvimento do sistema foi baseado em um processo de desenvolvimento ágil, onde em cada sprint eram realizadas as fases de análise, projeto, implementação, teste e implantação. Na fase de análise foi realizado o levantamento de requisitos a partir das informações contidas no manual de uso do sistema legado e complementadas pelos usuários do sistema. Na fase de projeto os modelos de análise foram refinados visando a construção ou adaptação da arquitetura do domínio. Os modelos da fase de projeto quase sempre eram informais, exceto em alguns casos mais complicados, com a intenção de dar uma visão geral do que deveria ser desenvolvido em determinado sprint. Na fase de implementação foram escritos os códigos necessários, juntamente com seus casos de teste. Ao fim do sprint era realizada a implantação no ambiente de homologação, permitindo a utilização por parte dos usuários e esperando seus feedbacks.

Durante a modelagem das entidades referentes ao domínio de Chamador, foi identificada a semelhança entre a definição de um chamador e uma regra do tipo ``se-então''. Como os chamadores servem de base para a geração de QDMs foi identificado que estes poderiam ser gerados automaticamente por um rule engine, a partir de um conjunto de regras ``se-então'', os chamadores. Foi então escolhido o Drools como rule engine, pois: a equipe já possuia conhecimento sobre a ferramenta, a existência de uma linguagem que permitiria a conversão, com facilidade, de chamadores para esta linguagem, vasta documentação e uma comunidade expressiva, além de ser desenvolvida por uma empresa de grande respeito no mundo do desenvolvimento, a RedHat, hoje parte da IBM.

Os testes iniciais para geração de QDMs exigiam a declaração de chamadores na linguagem Java, o que era, na maioria dos casos, muito moroso, exigindo dezenas de linhas de código, mesmo utilizando alguns métodos auxiliares que aumentavam o reuso. Além disso, a definição do resultado esperado, para comparar com o QDM gerado, necessitava da presença do usuário, para indicar os resultados reais. Quando não possível a presença do usuário o testador devia analisar manualmente o QC para identificar os resultados esperados, e declará-los em Java para a conferência no final do teste. Estes foram os fatores essenciais que levaram ao desenvolvimento de uma DSL, que visava facilitar a declaração de chamadores, e do uso do cucumber para a definição de casos de teste. Mas os testes ainda eram definidos de forma manual, embora já houvesse um ganho expressivo de produtividade, e ainda necessitavam do usuário ou de uma análise do QC para definição dos resultados esperados. Então foi desenvolvida uma solução para geração automática de testes para a geração de QDMs, onde eram gerados automaticamente tanto o chamador quanto o resultado esperado.


