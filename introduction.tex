\section{Introduction}\label{sec:Introduction}

Planning and distributing equipment to an army force is a
  challenging task, which involves several steps. First, the
  domain experts must specify all material that is relevant
  for a given class (for instance, operational material, official army clothing,
  and army ammunition). Second, domain experts must specify association 
rules that relate military materials to abstract 
organizational unities (e.g., departments, regiments,
and military positions). Third, the domain experts should 
execute the association rules for computing a list of expected
equipment for the \emph{abstract military unit}; and the
designation of the materials that must be
actually distributed to the \emph{concrete military organizations}. Concrete and
abstract military unities often differ because the first may suffer
variations due to the inclusion and suppression of positions
for its particular type of abstract organizational unit.

Decision support systems are essential in this domain, given the importance and complexity of the processes related to the allocation of military employment materials. In particular, the rules used to predict military materials and equipment are not trivial, and implementing these rules directly in the source code makes the systems difficult to understand and maintain. This problem mostly occurs because the number of conditions that need to be expressed is proportional to the complexity of the organizational structure and the number of materials. In the case of an entire army force (such as the Brazilian Army), this number is noteworthy. In addition, a ``\emph{hard-coding}'' solution does not allow domain experts to trivially specify and test new derivation rules. For these reasons, it is preferably to implement these rules declaratively, using languages, tools, or libraries that
support inference mechanisms.

Nevertheless, introducing either a logic language (e.g., Prolog or Datalog) or a specific library (such as Drools~\cite{browne2009}) in a complex enterprise
system might be a challenge task---particularly in environments whose whole ecosystem is well established and based on a set of constraints related to both language and libraries usage. Furthermore, these solutions require a steep learning curve, and thus it is worth to modularize and abstract their usage in enterprise systems.

In a previous work we describe the main design and architectural decisions we had
taken to implement the mechanisms that assist decision makers to plan the distribution of military materials and equipment across the organizational structure of the Brazilian Army~\cite{phtcosta:sbcars}. These decisions aim to both (a) abstract the adoption of a business rule engine using {\bf meta-programming} and (b) allow domain experts to simulate the definition and testing of rules using a {\bf domain specific language} (DSL). Altogether, previously we contributed with:

\begin{enumerate}[(a)] 

\item A description of the use of generative programming techniques to raise the abstraction level related to the adoption of a logic programming language in the specification of rules for the 	distribution of materials and equipment through organizational unities.

\item A report of an empirical evaluation of our design and architectural decisions, reinforcing that they fulfill not only the needs of the domain specialists, but also existing technical constraints. 	

\end{enumerate}


In this paper we extend our previous work with a
more detailed description of our DSL and an in depth evaluation of
our approach, considering a discussion about the results 
of a property based testing execution and a qualitative assessment
using the focus group method considering the domain experts responsible
for planing the distribution of operational equipment in the
Brazilian Army. 

{\color{red}Although we discuss the distribution of materials and equipment for the military domain (which is far from trivial), we believe that the architectural decisions discussed here can be exploited to other logistics and endorsements scenarios in which it is also desirable to transparently introduce the support for rule based engines in enterprise systems.}


